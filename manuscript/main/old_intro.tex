\section{Introduction}\label{intro}

Model simulations are key to understanding how climate impacts ecosystems over time. As the demand for reliable projections is increasing, the systematic evaluation of model performance should be one of the main concerns of modellers. Such evaluation remains critical to build confidence in model projections, and plays a crucial role in providing credible information for decision makers and stakeholders \cite{Dawson2011, Mouquet2015}.

Direct testing of the accuracy of model future projections is infeasible as the unforeseen will always remain so. Hence, the most straightforward approach to evaluate model reliability is to compare their predictions to observations from past periods as reproducing the past (\emph{hindcast performance/skill}) can be seen as a requisite condition to reliably forecast the future (\emph{forecast reliability}). While exact matches to expected 21\textsuperscript{st}-century climatic conditions do not exist in historical records \cite{Burke2018}, hindcasting exercises can nevertheless provide insights whether the models capture, implicitly or explicitly, the essential processes required to provide reliable projections in different conditions from the present.

In this regard, recent past observations (typically spanning a few decades) have been used for evaluating species distribution model predictions \cite{Araujo2005, Kharouba2009, Smith2013, Illan2014}. However, as these hincasts are made in a limited climate range similar to calibration conditions (present conditions), they do not provide a fully independent evaluation of the models. By looking much further back in the past, paleo-archives offer a unique framework to understand long-term climate-biodiversity dynamics \cite{Fordham2020} and to test species distribution model transferability in more challenging conditions, in the same way as climate models are evaluated using palaeoclimate proxies \cite{Braconnot2012}. Simulations in distant past, i.e. spanning several millennia, allow for model evaluation under conditions significantly different from present \cite{Maguire2015}, where climate variations were larger than those encountered during the last century.
Some studies have tested the transferability of species distribution models using paleoclimate reconstructions and fossil records \cite{Veloz2012, Pearman2008, Williams2013, Roberts2012}. They consistently reveal a decrease of the ability of the models to simulate accurate species distributions the more they move back to the past. This is not surprising as this is a common feature of models predicting the earth system, either the atmosphere (REF modele climatique) or the biosphere (REF DGVMs ?), but this calls nevertheless for caution when interpreting their projections in climatic conditions that differ significantly from the present \cite{Maguire2016}. And this is all the more important since no-analogue climatic conditions are forecasted to become more common \cite{Williams2007}, potentially compromising the reliability of model projections \cite{Fitzpatrick2018}. 

JE PENSE QU'IL FAUDRAIT ANNONCER NOTE OBJECTIF GENERAL ICI



\begin{figure}[ht]
\centering
\includegraphics[scale=0.8]{paleoclimate_overview.pdf}
\caption{Caption text}\label{paleoclimate_overview}
\end{figure}

\begin{comment}
While these investigations have yielded valuable insights into the reliability of species distribution models, they have primarily focused on correlative models, despite the growing interest for process-based models in predictive ecology \cite{Connolly2017, Urban2016, Pilowsky2022}.

This omission represents a notable gap in our understanding of the tenets of species distribution modelling, as only one side of the continuum between statistical and mechanistic approaches has been explored \cite{Dormann2012}, neglecting the investigation of process-based model performance. 
\end{comment}

To deal this issue, there is an expectation that including more biological realism and \emph{a priori} knowledge in the models hold the potential to enhance our ability to predict species responses to climate change and to provide more robust projections in novel conditions \cite{Evans2012, Singer2016}. Yet it is still unclear what conveys robustness to models, and whether these more realistic models (called process-based models) are truly more reliable. To test this hypothesis and bridge the gap in our understanding of the tenets of species distribution modelling, we here... multiple correlative and process-based models...

\begin{comment} 
There exist a variety of models used to predict the geographical distribution of species, from correlative models to more mechanistic models called process-based (PB) models \cite{Dormann2012}. 
\end{comment}
Correlative models infer statistical relationships between observations of species occurrences and environmental predictors. PB models are built upon explicit causal relationships determined experimentally between physiological, ecological and demographic processes and environmental drivers. Projections of PB models in future climatic conditions are systematically more conservative than those of correlative models \cite{Morin2009, Cheaib2012, Gritti2013}. However, despite the growing interest for process-based models in predictive ecology \cite{Connolly2017, Urban2016, Pilowsky2022}, very few studies have gone beyond qualitative comparisons between the models and compared thoroughly their performance, for example using virtual species \cite{Zurell2016}, exotic species in native and newly colonized areas \cite{Higgins2020}, or in the recent past \cite{Fordham2018}. While PB models have shown their usefulness for paleoecological studies \cite{Saltre2013, Ruosch2016, Schwoerer2014}, the extent to which they can provide more reliable predictions than correlative models in very different climatic conditions from present remains unknown as well as the reasons why they could do so \cite{UribeRivera2022, Briscoe2019}.
\begin{comment} 
Bridging this gap through a comprehensive evaluation of the different class of models is crucial \cite{Evans2016}, given that process-based models hold the potential to enhance our ability to predict species responses to climate change and to provide more robust projections in novel conditions \cite{Evans2012, Singer2016}. 
\end{comment}

To address these shortcomings, we propose a state-of-the-art comparison of the performance of correlative models and PB models to simulate paleodistributions of emblematic tree species of Europe at a high-temporal resolution. Long-term retrospective data indeed offer a great opportunity to determine what conveys robustness to models, which modelling approach yield the greates potential to provide the most reliable projections to guide climate adaptation strategies \cite{Fordham2016}. In order to  explore the differences between models, we used different versions of the models that differ by their level of complexity and the methods of estimation of their parameters. In particular, as hybrid models have been raised as a potential avenue \cite{Higgins2012, Dormann2012, Evans2016}, we use inverse modelling to fit PB models in the same way as the correlative models \cite{VanderMeersch2023}, to borrow strength from both approaches. Encompassing the spectrum of models, from correlative models to process-based models and their hybrid data-driven counterparts, allowed us identifying the key features necessary for building robust  models.