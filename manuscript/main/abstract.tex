
% originial
The recent acceleration of global climate warming has created an urgent need for reliable  ecological forecasts, especially for species distributions, which are widely used by natural resource managers. Such forecasts, however, are produced using varied modeling approaches with little information on which perform well given the novel climatic conditions expected with continued anthropogenic greenhouse gas emission. Here, we leverage the most recent period of climate novelty -- the early Holocene -- to compare the performance of three types of species distribution models and identify the origin of their robustness: correlative models,  process-based models and process-based models calibrated like correlative ones (inverse calibration). We hindcast the range shift of five forest tree species across Europe over the last 12,000 years, taking into account migration, and evaluate their predictions against fossil pollen records. While projections of all models are less reliable moving back to the past, we show that the  performance of correlative models decreases \textappr2.5 times faster than that of process-based models when climatic dissimilarity rises. We further find that inverse calibration of process-based models does not affect their reliability, and that process-based models should be more reliable than correlative models at least up to 2060 according to the scenario SSP245. These findings point  to the major importance of describing biological mechanisms to ensure model robustness, and highlight a new avenue to improve model projections in the future.


% premier essai 178 mots
Global climate warming has created an urgent need for reliable  ecological forecasts, especially for species distributions, which are widely used by ecosystem managers. Such forecasts, however, are produced using various modeling approaches with little information on which perform well given the novel climatic conditions expected.
We leverage the most recent period of climate novelty -- the early Holocene -- to compare the performance of several species distribution models and identify the origin of their robustness. 
We hindcast the range shift of five forest tree species across Europe and evaluate their predictions against fossil pollen records. While all models are less reliable moving back to the past, we show that the  performance of correlative models (CSDMs) decreases \textappr2.5 times faster than that of process-based models (PBMs) when climatic dissimilarity rises. We further find that inverse calibration of PBMs does not affect their reliability, and that they should be more reliable than CSDMs at least up to 2060. 
These findings point to the major importance of describing biological mechanisms to ensure model robustness, and highlight a new avenue to improve future projections.


% second essai 160 mots
Global climate warming has created an urgent need for reliable  ecological forecasts, especially for species distributions, widely used by ecosystem managers. Such forecasts, however, are produced using various modeling approaches with little information on which perform well given novel climatic conditions expected.
By leveraging the most recent period of climate novelty -- the early Holocene -- we assess the ability of several species distribution models to hindcast the range shift of five forest tree species across Europe. While all models are less reliable moving back to the past, results show that the performance of correlative models (CSDMs) decreases \textappr2.5 times faster than that of process-based models (PBMs) when climatic dissimilarity rises. We further find that inverse calibration of PBMs does not affect their reliability, and that they should be more reliable than CSDMs at least up to 2060. 
These findings point to the major importance of describing biological mechanisms to ensure model robustness, and highlight a new avenue to improve future projections.


% troisième essai 150 mots !!!
Global climate warming has created an urgent need for reliable  species distribution forecasts, widely used by ecosystem managers. Such forecasts, however, are produced using various models with little information on which perform well given expected novel climates.
By leveraging the most recent period of climate novelty -- the early Holocene -- we assess the ability of several species distribution models to hindcast five tree species range shifts across Europe. While all models are less reliable moving back to the past, results show that correlative models (CSDMs) performance decreases \textappr2.5 times faster than that of process-based models (PBMs) when climatic dissimilarity rises. We further find that inverse calibration of PBMs does not affect their reliability, and that PBMs should be more reliable than CSDMs at least up to 2060. 
These findings point to the major importance of describing biological mechanisms to ensure model robustness, and highlight a new avenue to improve future projections.

% version Isabelle
The recent acceleration of global climate warming has created an urgent need for reliable ecological forecasts. Such forecasts, however, are produced using varied modeling approaches with little information on which perform well given the novel climatic conditions expected by the end of the century. Here, we hindcast the range shift of five forest tree species across Europe over the last 12,000 years to compare the performance of three types of species distribution models and identify the origin of their robustness. We show that the performance of correlative models decreases 2.5 times faster than that of process-based models when climatic dissimilarity rises, and that process-based models should be more reliable than correlative models at least up to 2060 according to the scenario SSP245. These findings point to the major importance of describing biological mechanisms to ensure model robustness, and highlight a new avenue to improve model projections in the future.


% Version Isa modifiée - 149 mots
The recent acceleration of global climate warming has created an urgent need for reliable ecological forecasts, especially for species distributions. Such forecasts, however, are produced using various modeling approaches with little information on which perform well given the novel climatic conditions expected by 2100. Here, we hindcast the range shift of five forest tree species across Europe over the last 12,000 years to compare the performance of three types of species distribution models and identify the origin of their robustness. We show that the performance of correlative models (CSDMs) decreases 2.5 times faster than that of process-based models (PBMs) when climatic dissimilarity rises, and that PBMs should be more reliable than CSDMs at least up to 2060 according to the scenario SSP245. These findings point to the major importance of describing biological mechanisms to ensure model robustness, and highlight a new avenue to improve model projections in the future.